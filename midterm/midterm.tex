\documentclass[11pt]{extarticle}
\usepackage[utf8]{inputenc}
\usepackage{cite, authblk, mdframed}
\usepackage[margin=1in]{geometry}

\title{Project: Midterm Report}
\author[1]{Evan Matthews}
\author[1]{Vikram Ramavarapu}
\author[1]{Krishnaveni Unnikrishnan}
\affil[1]{CS 412 Group G6}
\date{November 6th, 2024}

\newcommand{\todo}{\textcolor{red}{TODO:}~}

\begin{document}

\maketitle

\pagebreak

% \section{Title} 
% starting with group id, please also include names of group members

\section{Abstract}
summarizing the project\cite{Pettorruso2020-qt,Cash2012-rb,Aboujaoude2010-mc,Restrepo2020-pb}.

\begin{mdframed}
\end{mdframed}

\section{Introduction}

The internet has become an integral part of our daily lives, with people of all ages spending a significant amount of time online. 
This trend has given rise to concerns about the potential impacts of excessive internet use, particularly on children and teens.
Problematic Internet Use (PIU) is a condition characterized by excessive or poorly controlled preoccupations, urges, or behaviors regarding computer use and internet access that lead to impairment or distress \cite{Pettorruso2020-qt}. 
PIU has been associated with a range of mental health issues, including depression, anxiety, and impulsivity \cite{Cash2012-rb}.
As such, identifying early signs of PIU in children and teens is crucial for prevention and intervention.
In this project, we aim to predict early signs of PIU in children and teens using machine learning techniques, leveraging data from the Child Mind Institute's Healthy Brain Network.
The project plan consists of three phases: data preprocessing, initial model evaluation, and fine-feature reevaluation.
We will submit our work to the Child Mind Institute's (CMI) Kaggle competition on PIU prediction, and we also aim to publish our results as a paper should they outperform competition expectations.


\section{Motivation} 
\todo A few sentences on why the project is of interest from a data mining and/or real world application perspective.

\begin{mdframed}
    With the rise of machine learning and pattern prediction models, the ability to analyze and predict upon more complex data and parameters becomes much more approachable.
    Likewise, child development is a multi-facted situation in which parenting and environmental factors can lead to an incredibly high number of outcomes.
    This field has had great strides in classical research, but a more modern approach could lead to significant development in the success of future generations.
    Additionally, predictions against an extensive number of possible outcomes like this represents a current roadblock in machine learning- that is, how modern predictive models can adapt to an ever-increasing set of parameters and decreasing set of training data.
    Finally, child psychology is interested in recognizing patterns in early behavior in order to reduce the impact of harmful effects from a child's environment.
\end{mdframed}

\section{Related work} 
in the literature, and in kaggle/related forum. Having just 1-2 references or no references to papers/books will lead to low scores. 
\todo Add references to related work

\section{scope} 
\todo Any change in scope from original proposal, please see guidelines above.

\begin{mdframed}
    Given that the original scope of the project was accepted, we are pressing forward with this plan with no significant changes.
    The most crucial critique provided- that the validation plan and evaluation metric were not clear- are likewise addressed in the methodology section.
\end{mdframed}

\section{Methodology} 
\todo what you are doing / planning to do from a data mining perspective. 
This can include any exploratory or statistical data analysis, visualization, efficient data storage/compression, fitting predictive models, clustering, pattern mining, etc.  
If you are doing / planning to do a comparative study, discuss the methods you are considering, including any methods being used as baselines.  
Please include a clear evaluation plan, including train-val-test splits, cross-validation, hyper-parameter tuning, and explain clearly how you will do these. 
You have to highlight alignment of the project with the course clearly, especially if you have received feedback on concerns regarding limited alignment with the course. 

\begin{mdframed}
    Add methodology here
\end{mdframed}

\section{(Current / Preliminary) Results} 
\todo what you have so far in terms of initial results and analysis of initial results. Please see comment on figures/tables above, especially the fact that good captions go a long way to making things readable.

\begin{mdframed}
    Add results here
\end{mdframed}

\section{Plan of Work} 
\todo what are the next steps before the final report. Please be as precise as possible. Note that you will have about a month to finish the project, so make suitably calibrated plans, e.g., do not over/under promise.

\begin{mdframed}
    Add plan of work here
\end{mdframed}

\section{Conclusions, discussions}

\begin{mdframed}
    add conclusions here
\end{mdframed}

\bibliographystyle{plain}
\bibliography{bibliography}

\end{document}