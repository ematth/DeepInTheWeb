\documentclass[12pt]{extarticle}
\usepackage[utf8]{inputenc}
\usepackage[margin=1in]{geometry}
\usepackage[title]{appendix}
\usepackage{graphicx, listings, titlesec, cite, authblk, mdframed, floatrow}

\titleformat{\section}
  {\normalfont\fontsize{12}{15}\bfseries}{\thesection}{1em}{}

\title{Fun name for the project}
\author[1]{Evan Matthews}
\author[1]{Vikram Ramavarapu}
\author[1]{Krishnaveni Unnikrishnan}
\affil[1]{CS 412 Group G6}
\date{December 11, 2024}

% commands
\newcommand{\todo}{\textcolor{red}{TODO:}~}
\newfloatcommand{capbtabbox}{table}[][\FBwidth]

\begin{document}

\maketitle

\begin{abstract}
    The Internet's pervasive role in modern life has raised concerns about Problematic Internet Use (PIU), particularly among children and teens. 
    Our research aims to predict early signs of PIU using machine learning techniques applied to data from the Child Mind Institute's Healthy Brain Network. 
    This study employs a comprehensive methodology combining both cross-sectional and time-series data for future analysis. 
    Initial results from multiple models, including Random Forest, XGBoost, SVM, and Feed Forward Neural Networks, demonstrate promising accuracy rates, with XGBoost achieving the highest mean accuracy of 0.682. 
    Our project experimentation is structured in three phases: data preprocessing, initial model evaluation, and fine-feature reevaluation. 
    The methodology incorporates innovative approaches such as sequential modeling for time-series data and ensemble techniques combining cross-sectional and sequential models. 
    Preliminary findings suggest that machine learning can effectively predict PIU severity using quantitative measures compared to traditional assessments. 
    This research contributes to the growing field of digital health by providing a data-driven approach to identifying at-risk youth for PIU.
\end{abstract}

\pagebreak

\section{Introduction}

\section{Motivation}

\section{Related Work}

\section{Methodology}

\section{Empirical Results}

\section{Discussion}

\section{Conclusion}

\pagebreak
\nocite{*}
\bibliographystyle{plain}
\bibliography{bibliography}

\pagebreak

\begin{appendices}
    \section*{Appendix}
    
    \section{Data Preprocessing}
    \section{Model Evaluation}
    \section{Feature Reevaluation}
    \section{Code}
  
\end{appendices}

\end{document}