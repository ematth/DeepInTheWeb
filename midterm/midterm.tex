\documentclass[11pt]{extarticle}
\usepackage[utf8]{inputenc}
\usepackage{cite, authblk, mdframed}
\usepackage[margin=1in]{geometry}

\title{Project: Midterm Report}
\author[1]{Evan Matthews}
\author[1]{Vikram Ramavarapu}
\author[1]{Krishnaveni Unnikrishnan}
\affil[1]{CS 412 Group G6}
\date{November 6th, 2024}

\newcommand{\todo}{\textcolor{red}{TODO:}~}

\begin{document}

\maketitle

\pagebreak

% \section{Title} 
% starting with group id, please also include names of group members

\section{Abstract}
summarizing the project\cite{Pettorruso2020-qt,Cash2012-rb,Aboujaoude2010-mc,Restrepo2020-pb}.

AI-generated stuff:
\begin{mdframed}
    The project aims to develop a comprehensive data mining solution to analyze and predict trends in large datasets. 
    The primary focus is on leveraging advanced machine learning techniques to extract meaningful insights and patterns. 
    The project will involve data preprocessing, exploratory data analysis, model training, and evaluation. 
    The expected outcome is to build a robust predictive model that can be applied to real-world scenarios, providing valuable information for decision-making processes. 
    The project aligns with the course objectives by applying theoretical concepts to practical applications, demonstrating the effectiveness of data mining methodologies.
\end{mdframed}

\section{Introduction}
\todo a self-contained intro to the problem, including motivation 
(why the problem is interesting/important, etc.)

\section{Motivation} 
\todo A few sentences on why the project is of interest from a data mining and/or real world application perspective.

\section{Related work} 
in the literature, and in kaggle/related forum. Having just 1-2 references or no references to papers/books will lead to low scores. 
\todo Add references to related work

\subsection{scope} 
\todo Any change in scope from original proposal, please see guidelines above. 

\section{Methodology} 
\todo what you are doing / planning to do from a data mining perspective. This can include any exploratory or statistical data analysis, visualization, efficient data storage/compression, fitting predictive models, clustering, pattern mining, etc.  If you are doing / planning to do a comparative study, discuss the methods you are considering, including any methods being used as baselines.  Please include a clear evaluation plan, including train-val-test splits, cross-validation, hyper-parameter tuning, and explain clearly how you will do these. You have to highlight alignment of the project with the course clearly, especially if you have received feedback on concerns regarding limited alignment with the course. 

\section{(Current / Preliminary) Results} 
\todo what you have so far in terms of initial results and analysis of initial results. Please see comment on figures/tables above, especially the fact that good captions go a long way to making things readable.

\section{Plan of Work} 
\todo what are the next steps before the final report. Please be as precise as possible. Note that you will have about a month to finish the project, so make suitably calibrated plans, e.g., do not over/under promise.

\section{Conclusions, discussions}

\bibliographystyle{plain}
\bibliography{bibliography}

\end{document}