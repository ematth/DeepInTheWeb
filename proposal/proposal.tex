\documentclass[12pt]{extarticle}
\usepackage[utf8]{inputenc}
\usepackage{cite}

\title{CS 412 Project Proposal}
\author{Evan Matthews, Vikram Ramavarapu, Krishnaveni Unnikrishnan}
\date{Fall 2024}

\begin{document}

\maketitle

\section{Introduction}

The problem: We want to understand how we can use predictors such as physical activity, sleep, and duration of internet usage to determine the level of problematic internet usage among young populations.

Q: What is problematic internet use? 

\section{Motivation}

Problematic Internet Use (PIU), also known as Internet Addiction or Internet Use Disorder, refers to excessive or poorly controlled internet use that leads to negative consequences in a person's life. It can affect social, academic, occupational, and psychological functioning.

In the context of the Healthy Brain Network (HBN) study and the competition you mentioned, PIU would be assessed by looking at behaviors such as:

Excessive time spent online: Spending more time than intended or needed on internet-related activities.
Neglect of other responsibilities: Ignoring social, academic, or personal obligations in favor of internet use.
Social isolation: Reduced real-life interactions, leading to loneliness or social issues.
Emotional or mental health impact: Internet use becoming a coping mechanism for emotional distress, leading to anxiety, depression, or other mental health issues.
Inability to reduce use: Even when recognizing the negative impact, the individual may find it difficult to limit their time online.
The Severity Impairment Index (SII) in this case would likely quantify the degree to which internet use negatively affects the participants' lives, helping researchers better understand the link between internet behaviors and mental health outcomes.

\section{Data}

\section{Plan}

\bibliographystyle{plain}
\bibliography{proposal.bib}

\end{document}